\documentclass[12pt]{exam}
\usepackage{sinhala}

\usepackage[margin=1in]{geometry}
\usepackage{amsmath,amssymb}
\usepackage{multicol}

\newcommand{\class}{සංයුක්ත ගණිතය I}
\newcommand{\term}{දෙවැනි වාරය}
\newcommand{\examnum}{I කොටස}
\newcommand{\examdate}{\today}
\newcommand{\timelimit}{පෑ එකයි}

\pagestyle{head}
\firstpageheader{}{}{}
\runningheader{\class}{\examnum\ - පිටු \numpages\ න් \thepage}{\examdate}
\runningheadrule


\begin{document}

\noindent
\begin{tabular*}{\textwidth}{l @{\extracolsep{\fill}} r @{\extracolsep{6pt}} l}
\textbf{\class} & \textbf{නම:} & \makebox[2in]{\hrulefill}\\
\textbf{\term} & \textbf{පංතිය:} & \makebox[2in]{\hrulefill}\\
\textbf{\examnum} & \textbf{පංති අංකය:} & \makebox[2in]{\hrulefill}\\
\textbf{\examdate} &&\\
\textbf{කාලය: \timelimit} &&
\end{tabular*}\\
\rule[2ex]{\textwidth}{2pt}

මෙම ප්‍රශ්න පත්‍රය පිටු \numpages\ කින් (මෙම පිටුවද ඇතුළුව) හා ප්‍රශ්න \numquestions\ කින් සමන්විතයි.\\
ලබා ගත හැකි මුළු ලකුණු සංඛ්‍යාව \numpoints\ ක් වේ.

හැඳින්වීමේ ඉතිරි කොටස මෙතැනින් සම්පූර්ණ කරන්න. හැඳින්වීමේ ඉතිරි කොටස මෙතැනින් සම්පූර්ණ කරන්න. හැඳින්වීමේ ඉතිරි කොටස මෙතැනින් සම්පූර්ණ කරන්න.

\begin{center}
ලකුණු වගුව (ගුරුවරයාගේ භාවිතය සඳහා පමණි.)\\
\addpoints
\gradetable[v][questions]
\end{center}

\noindent
\rule[2ex]{\textwidth}{2pt}

\begin{questions}

\question[1] $2+2$ අගයන්න.
\addpoints

\question[20] $f(x)=3x^3+2x^2+x+1$ ශ්‍රිතය සලකන්න.
\noaddpoints % to omit double points count
\begin{parts}
\part[10] $f'(x)$ සොයන්න.
\part[10] $f''(x)$ සොයන්න.
\end{parts}
\addpoints

\question[2] නොගැළපෙන පිළිතුර තෝරන්න.
\begin{choices}
\choice නිමේශ්
\choice හිමේශ්
\choice ආකාෂ්
\choice සෝමරතන
\choice ජෙරාඩ්
\end{choices}

\question[2] නොගැළපෙන පිළිතුර තෝරන්න.
\begin{oneparchoices}
\choice නිමේශ්
\choice හිමේශ්
\choice ආකාෂ්
\choice සෝමරතන
\choice ජෙරාඩ්
\end{oneparchoices}

\question[3] නිවැරැදි පිළිතුරු සලකුණු කරන්න.
\addpoints
\begin{checkboxes}
\choice $2+2=4$
\choice $\frac{d}{dx} (x^2+1) = 2x+1$
\choice හඳට පොලු ගැසිය හැකිය.
\end{checkboxes}

{%
\checkboxchar{$\Box$} % changing checkbox style locally
\question[3] නිවැරැදි පිළිතුරු සලකුණු කරන්න.
\addpoints
\begin{checkboxes}
\choice $2+2=4$
\choice $\frac{d}{dx} (x^2+1) = 2x+1$
\choice හඳට පොලු ගැසිය හැකිය.
\end{checkboxes}
}%

{%
% changing choice items style locally
\renewcommand*\thechoice{\arabic{choice}} 
\renewcommand*\choicelabel{\thechoice)}
%
\question[2] පරමාණුක ක්‍රමාංකය 92 වන මූලද්‍රව්‍යය වන්නේ:
\begin{multicols}{2}
\begin{choices}
\choice H
\choice O
\choice F
\choice S
\choice Ba
\choice Pb
\choice U
\choice Pu
\end{choices}
\end{multicols}
}%

\question[10]
එක ඡේදයක් පමණක් භාවිත කරමින් පෘථිවිය ගෝලාකාර බැව් පැහැදිළි කරන්න.
\makeemptybox{2in}

\question[20]
සම්‍යක් ප්‍රවාද යනුවෙන් අදහස් කරන්නේ කුමක්දැ යි උදාහරණ සහිතව පැහැදිළි කරන්න.
\makeemptybox{\fill}

\newpage

\question[20]
අධ්‍යාහෘත ශ්‍රිතයක් යනු කුමක් ද?
\fillwithlines{\fill}

\newpage

\question[20]
අපාරගම්‍ය පටලයක් යනු කුමක් ද?
\fillwithdottedlines{8em}

\end{questions}

\end{document}
